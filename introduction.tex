\chapter{Uvod}

Stezosledec je domišljijska namizna igra vlog, kjer s prijatelji gradite pripoved o pogumnih junakih in zvitih hudobnežih v svetu, polnem strašnih pošasti in čudovitih zakladov. Še pomembneje, odločitve tvojega lika pomagajo usmerjati potek zgodbe.

Pustolovščine se odvijajo v Dobi razgubljenih znamenj, nevarnem domišljijskem svetu, ki je bogat s starodavnimi cesarstvi, velikimi mestnimi državami ter grobnicami, temnicami in pošastnimi brlogi, v katerih se skrivajo bleščeči zakladi. Liki lahko zaidejo v zapuščene podvodne razvaline, zaklete gotske kripte ali čarobne univerze v džungelskih mestih. Čakajo te brezštevilne pustolovščine!

\section{Kaj je igra vlog?}

Igra vlog je interaktivna pripoved, kjer eden izmed igralcev, Vodja igre, predstavi prizorišče in izzive, medtem ko ostali igralci prevzamejo vloge igranih likov in poskušajo premagati te izzive. Tekom igre naletijo na številne nevarnost, na primer pošasti, pasti in poseganja zlobnih nasprotnikov, vendar v Stezosledcu najdemo tudi politične spletke, uganke, medosebne odnose in še veliko več.

Igro praviloma igra med štiri in sedem igralcev, kar vključuje Vodjo igre. Vodja igre pripravi, predstavi ter nadzoruje zgodbo in svet, v katerem se igra odvija, nastavlja ovire igralcem in igra nasprotnike, zaveznike in ostale osebe. Tekom igre vsi igralci pomagajo splesti zgodbo in se odzivajo na dane okoliščine skladno z osebnostmi in sposobnostmi svojih likov. Meti kocke skupaj z vnaprej dodeljenimi vrednostmi prispevajo element naključja in določajo, ali likom njihova dejanja uspejo ali spodletijo.

\begin{rpg-titlebox}{Prvo pravilo}
    Prvo pravilo Stezosledca je, da je igra povsem v tvojih rokah. Zgodbe spleti po lastnih željah. Igraj like, ki so ti všeč. Deli vznemirljive pustolovščine s svojimi prijatelji. Če vam kdaj kakšno pravilo pokvari zabavo, ga lahko spremenite ali ga zanemarite, dokler se seveda vsi strinjajo. Glavni cilj Stezosledca je, da se vsi zabavajo.
\end{rpg-titlebox}

\begin{rpg-titlebox}{Kocke}
    \label{sidenote-kocke}
    Za igro potrebuješ komplet večstranih kock. Vsaka izmed njih ima različno število ploskev -- štiri, šest, osem ali več. V besedilu te kocke označujemo s črko „k“, ki ji sledi število plati kocke. Za Stezosledca boš potreboval/a štiristrano (k4), šeststrano (k6), osemstrano (k8), desetstrano (k10), dvanajststrano (k12) in dvajsetstrano kocko (k20). Če moraš vreči več kock iste vrste, število pred črko „k“ pove, koliko. Na primer, zapis „4k6“ pomeni, da moraš vreči štiri šeststrane kocke. Zapis k\% označuje naključno število med 1 in 100, ki pa ga dobimo tako, da vržemo dve desetstrani kocki, pri čemer ena predstavlja enice, druga pa desetice.
\end{rpg-titlebox}

\subsection{Potek igre}

Igralci se pred pričetkom igre zberejo, bodisi v živo bodisi na daljavo prek interneta. Tako srečanje navadno traja nekaj ur. Zaključeno pripoved je mogoče odigrati v enem srečanju, tako odigrani pustolovščini pa pravimo enopopoldnevnica/enovečernica (s tujimi besedami „\textit{one-shot}“). Ena sama zgodba se lahko odvija skozi več srečanj in traja nekaj mesecev ali let. Taki pustolovščini rečemo kampanja. Če Vodja igre uživa v pripovedovanju zgodbe, igralci pa se zabavajo, lahko igra traja poljubno dolgo.

Potek posameznega srečanja lahko zaznamuje napeta akcija, denimo bitke s strašnimi pošastmi, reševanje iz peklenskih pasti in izpolnjevanje junaških del. Po drugi strani se lahko srečanje vrti okoli pogajanja z baronom za pravice do utrdbe, vtihotapljanja v vojsko ledenih velikanov ali barantanja z angelom za pramen njegovih las, s katerim bi oživili vašega pobitega prijatelja. Vaša igralna skupina določa, kakšno vrsto igre boste igrali -- od raziskovanja temnic do prodorne politične drame, pa še karkoli vmes.

\subsection{Igralci}

Igralci so vsi, ki igrajo Stezosledca, vključno z Vodjo igre, vendar zavoljo enostavnosti „igralec“ običajno pomeni vse druge udeležence razen Vodje. Preden se igra prične, si igralci za svoje like izmislijo življenjsko zgodbo in osebnost. S pomočjo pravil določijo igralne vrednosti, sposobnosti, prednosti in slabosti svojih likov. Vodja igre lahko omeji možnosti, ki so jim na voljo pri ustvarjanju likov, vendar so te omejitve odločene že vnaprej, da lahko vsi ustvarijo zanimive junake. V splošnem izdelavo likov omejujejo le igralčeva domišljija in Vodjeva pravila.

Med igro igralci opisujejo, kaj njihovi liki počno, in mečejo kocke v kombinaciji z sposobnostmi njihovih likov. Vodja določi izid teh dejanj. Nekateri igralci se radi vživijo v svoje like in jih odigrajo (temu rečemo igranje vloge), drugi pa raje opisujejo dejanja svojih likov kot pripovedovalci zgodbe. Izberi tisto, kar ti bolj ugaja! Če se prvič srečuješ z igro vlog, predlagamo, da začneš v vlogi igralca, da se lažje spoznaš s pravili in svetom igre.

\subsection{Vodja igre}

Medtem ko drugi igralci ustvarijo in igrajo svoje like, Vodja igre krmili zgodbo in svet, v katerem se le-ta dogaja. Vodja opiše vse situacije, v katerih se igralski liki znajdejo, premisli, kako dejanja igralcev vplivajo na pripoved, in uveljavlja pravila.

Vodja igre lahko ustvari novo pustolovščino -- stke zgodbo, izbere pošasti, postavlja zaklade -- ali pa izbere že pripravljeno pustolovščino, ki jo priredi za potrebe igralcev in načina igranje skupine. Nekateri Vodje celo kombinirajo lastno iznajdbe in že objavljen material ter tako ustvarijo edinstveno zgodbo.

Vloga Vodje igre ni enostavna, saj zahteva, da sodiš po pravilih, pripoveduješ zgodbo in izpolnjuješ še druge obveznosti. A hkrati ti da tudi velik občutek zadovoljstva, ko se ves trud končno izplača. Če prvič vodiš igro, si zapomni, da je važno samo to, da se vsi zabavajo, vključno s tabo. Vse ostalo pride s časom in vajo.

\subsection{Igranje je za vsakogar}

Najsi bo Vodja igre ali igralec, igranje igre vlog od vseh zahteva medsebojni dogovor: vsi ste se zbrali, da bi se imeli fino in spletli skupno pripoved. Mnogim igranje vlog predstavlja pobeg pred vsakdanjikom. Upoštevaj ostale igralce in poskrbi, da se boste vsi zabavali. Ko se prvič sestanete za mizo, se pogovorite, kakšno izkušnjo vsak pričakuje in česa naj se izogibate. Vsi morajo razumeti, da utegne priti do trenutkov, ko bi se kakšen igralec nemara počutil neprijetno ali celo nezaželenega, in vsi se morajo med igro držati teh mej. Na ta način lahko vsi uživajo.

Stezosledec je igra za vse starosti, spole, rase, etnična ozadja, verske nazore, spolne usmerjenosti ... Vsi, ne samo Vodja igre, morajo poskrbeti, da bo vzdušje za mizo prijetno.

\subsection{Orodja za igranje}

Poleg te knjižice je še nekaj reči, ki jih potrebuješ za igro. Najdeš jih lahko v bolje založeni trgovini z družabnimi igrami ali na spletišču paizo.com.

\textbf{Pustolovski obrazec:} Vsak igralec bo za ustvarjanje svojega lika in za beleženje njegovega stanja med igro potreboval pustolovski obrazec. Na voljo je na koncu te knjižice ali brezplačno na spletu v formatu PDF.

\textbf{Kocke:} Igralci in Vodja boste potrebovali vsaj en komplet večstranih kock, čeprav je v navadi, da večina sodelujočih prinese svoj komplet. Šeststrane kocke so precej običajno, ostale pa je mogoče najti v bolje založenih trgovinah z družabnimi igrami ali na spletu. Preberi si pripis Kocke na \pageref{sidenote-kocke}. strani, kjer so navedene vrste kock in njihova navedba v besedilu.

\textbf{Pustolovščina:} Za igranje igre potrebujete pustolovščino. To lahko ustvari Vodja sam ali pa uporabi že izdano. Na spletišču paizo.com lahko najdeš pester nabor zanimivih pustolovščin in celih kampanj.

\textbf{Enciklopedija zveri:} Od strašnih zmajev do nagajivih gremlinov -- pošasti so običajna nevarnost, na katero lahko igrani liki naletijo. Vsaka pošast ima čisto svoje lastnosti in veščine, ki jih lahko najdeš v knjigi \textit{Stezosledčeva enciklopedija zveri} (angleško \textit{Pathfinder Bestiary}), ki je za Vodjo igre nepogrešljiva. Brezplačno lahko do lastnosti pošasti dostopaš tudi na paizo.com/prd.

\textbf{Zemljevidi in figurice:} Zmedo med spopadom si je lahko težko osmisliti, zato številne skupine bojno polje predstavijo z zemljevidi. Ti so označeni z kvadratasto mrežo, kjer vsak kvadratek predstavlja razdaljo 1,5 metra v igri. S figuricami in pobarvanimi žetoni predstavimo like in njihove nasprotnike.

\textbf{Drugi pripomočki:} Za zanimivejšo ali udobnejšo igro si lahko omisliš še druge pripomočke, denimo orodja, ki ti pomagajo beležiti poteze med spopadom, kartice z najpogosteje rabljenimi pravili, digitalna orodja za ustvarjanje likov ter predvajanje glasbe in zvočnih učinkov med igro.

\section{Osnove igre}

Preden ustvariš svoj prvi lik ali pustolovščino, moraš najprej razumeti nekaj osnovnih idej, ki se uporabljajo v igri. Novi pojmi bodo v besedilu odebeljeni, da jih bo lažje najti, vendar je to poglavje samo uvod v osnove igre. Vsa igralna pravila so nanizana v kasnejših poglavjih.

\subsection{Kaj vse predstavlja lik}

V igri Stezosledca igralci zasedejo vlogo \textbf{igranih likov}, dočim Vodja igre upravlja z \textbf{neigranimi liki} in \textbf{pošastmi}. Tako igrani kot neigrani liki so pomembni za zgodbo, vendar služijo povsem različnim namenom. Igrani liki so protagonisti, se pravi se zgodba vrti okoli njih, neigrani liki in pošasti pa so njihovi zavezniki, znanci, nasprotniki in hudobneži. Kljub temu imajo vsi liki in pošasti nekatere skupne lastnosti.

\textbf{Stopnja} je eden najpomembnejših atributov v igri, saj označuje približno moč in zmožnosti vsakega posameznega bitja. Stopnje igranih likov sežejo od prve, s katero lik prične svojo pustolovsko pot, do dvajsete, ko je lik na vrhuncu svoje moči. Ko liki premagajo izzive in sovražnike ter končajo pustolovščine, prejmejo \textbf{točke izkušenosti (IT)}. Kadarkoli lik zbere 1000 točk izkušenosti, se povzpne za eno stopnjo višje in prejme nove sposobnosti, s katerimi se lahko poloti težjih izzivov. Igrani lik prve stopnje se lahko postavi po robu orjaški podgani ali skupini tolovajev, toda na dvajseti stopnji lahko isti lik poruši cela mesta z enim samim urokom.

Poleg stopnje imajo liki tudi \textbf{ocene sposobnosti}, ki merijo likov surovi potencial in s katerimi izračunamo večino ostalih igralnih parametrov. Poznamo šest ocen sposobnosti. \textbf{Moč} predstavlja likovo telesno krepkost, medtem ko \textbf{Gibčnost} predstavlja okretnost in sposobnost izogibanja nevarnosti. \textbf{Konstitucija} določa likovo zdravje in počutje. \textbf{Inteligenca} predstavlja čisto znanje in sposobnost reševanja miselnih ugank, \textbf{Modrost} pa ocenjuje likovo dojemljivost in sposobnost ocenjevanja situacije. Zadnja je \textbf{Karizma}, ki vključuje šarm, prepričljivost in moč osebnosti. Vrednosti ocen sposobnosti pri navadnih posameznikih segajo od 3 do 18, pri čemer 10 predstavlja povprečne človeške zmožnosti. Pri likih višje stopnje lahko vrednosti ocen sposobnosti zrasejo tudi precej čez 18.

Ocena sposobnosti, ki je višja od povprečja, poveča tvoje možnosti za uspeh pri opravilih, povezanih s to sposobnostjo. Ocena sposobnosti, nižja od povprečja, verjetnost za uspeh zmanjša. Vrednosti, ki spreminja verjetnost uspeha, pravimo \textbf{popravek sposobnosti}.

Svojemu igranemu liku določiš tudi nekaj drugih atributov. Najprej izbereš likov \textbf{rod}, ki predstavlja likove starše in dediščino, na primer človeško, vilinsko ali goblinsko. Sledi likovo \textbf{ozadje}, ki opisuje njegovo odraščanje in zgodnje življenje. Nazadnje likov \textbf{razred} določi njegovo nadarjenost na nekaterih področjih in sposobnosti, na primer čaranje mogočnih urokov pri čarovniku in moč spreminjanja v strašne zveri pri druidu.

Poleg tega imajo igrani liki tudi nekaj \textbf{posebnih izučitev} -- posameznih sposobnosti, ki jih izbereš med ustvarjanjem lika in na prehodih z ene stopnje na naslednjo. Posebne izučitve delimo na več vrst, da jih je lažje najti (vilinske izučitve se na primer nahajajo pri informacijah o vilinskem rodu), in vsaka ima svojo tematiko (čarovniške izučitve zajemajo sposobnosti, ki imajo opravka s čaranjem). Naposled imajo liki tudi \textbf{veščine}, ki merijo njihove zmožnosti skrivanja, plavanja, barantanja in drugih običajnih opravil.

\subsection{Ustvarjanje zgodbe}

Zgodbo Stezosledca ustvarjajo liki s svojimi odločitvami, a kako vplivajo drug na drugega in na okolico, urejajo pravila. In čeprav se odločiš, da se bo tvoj lik podal na dolgo potovanje, da bi premagal strašne sovražnike in rešil svet, so njegove možnosti za uspeh odvisne od njegovih sposobnosti, tvojih odločitev in kock.

Vodja igre določi osnovno vodilo zgodbe in ozadje večine pustolovščin, čeprav sta pogosto pomembni tudi preteklost in osebnost likov. Ko se srečanje prične, igralci po vrsti opisujejo namere svojih likov, Vodja pa določa njihove izide. Poleg tega opisuje tudi okolico, dejanja neigranih likov in druge dogodke. Na primer, Vodja igre lahko naznani, da so like in njihovo domačo vas napadli plenilski troli. Liki lahko trole preženejo v bližnjo močvaro -- tedaj pa se izkaže, da so troli pobegnili iz svoje močvare zaradi prihoda groznega zmaja! Liki imajo tako na izbiro boj s troljim plemenom, z zmajem ali z obema. Ne glede na svojo odločitev njihov uspeh temelji na njihovih izbirah in metanju kock med igro.

Zaokrožena zgodba -- vključno z zapletom, osrednjim dogajanjem in zaključkom -- se imenuje \textbf{pustolovščina}. Zaporedje pustolovščin splete še obširnejšo pripoved, ki ji rečemo \textbf{kampanja}. Za dokončanje pustolovščine je lahko potrebnih tudi več srečanj, kampanja pa lahko traja mesece ali leta!

\begin{rpg-titlebox}{Svet kot del igre}
    Poleg likov in pošasti lahko tudi sam svet Stezosledca vpliva na pripoved. Svet lahko deluje na jasen in viden način, na primer da po deželi divjajo hude nevihte, lahko pa je tudi bolj nevpadljiv. V mnogih pripovedkah so pasti in zakladi prav tako pomembni kot krvoločne zveri. Da ti bo lažje razumeti vse te elemente igre, imajo številni med njimi enake karakteristike kot liki in pošasti. Na primer, večina okoljskih nevarnosti ima stopnjo, ki označuje njihovo resnost, stopnja magičnega predmeta pa predstavlja njegovo moč in pomembnost za zgodbo.
\end{rpg-titlebox}

\section{Igranje igre}

V igri Stezosledca vsak prizor zaznamujejo trije igralni segmenti. Večino časa tvoj lik \textbf{raziskuje}, odkriva skrivnosti, rešuje težave in se sporazumeva z drugimi liki. Toda Doba razgubljenih znamenj ne varčuje z nevarnostjo, tako da liki pogosto \textbf{naletijo} na brutalne zveri in krvoločne pošasti. Liki prej ali slej dočakajo tudi \textbf{prosti čas}, ko si lahko oddahnejo od groženj in tveganj ter se odpočijejo in pripravijo na prihodnje odprave. Skozi pustolovščino prehajamo prek teh treh igralnih segmentov. Dlje kot boš igral/a, jasneje ti bo postalo, kakšne so razlike med segmenti. Toda prehajanje iz enega segmenta v drugega ima nekaj jasnih omejitev.

Med igro se bo tvoj lik znašel v položaju, ko izid ne bo gotov. Nemara bo moral preplezati prepadno steno, izslediti ranjeno himero ali se splaziti mimo spečega zmaja, kar so nevarna opravila, kjer plačaš visoko ceno za neuspeh. V takih primerih bo moral lik (ali več likov) izvesti \textbf{preizkus}, s katerim bo preveril, ali mu je uspelo ali ne. Preizkus se običajno opravi z metom dvajsetstrane kocke (k20), k njegovemu izidu pa prištejemo neko številko, ki ustreza sposobnosti, v kateri se preizkušamo. V takem primeru je višji izid meta vedno ugodnejši.

Po metu za preizkus Vodja primerja rezultat meta s ciljnim številom, imenovanim \textbf{težavnostni razred (TR)}. Če je rezultat preizkusa večji ali enak težavnostnemu razredu, preizkus uspe. Če je manj, preizkus spodleti. Če težavnostni razred premagamo za 10 ali več, temu rečemo \textbf{skrajni uspeh} in običajno predstavlja posebej ugoden izid. Če nam do težavnostnega razreda zmanjka 10 ali več, to pomeni \textbf{skrajni neuspeh} in včasih privede do dodatnih negativnih posledic. Pogosto skrajni uspeh dosežeš tudi, če pri metu za preizkus vržeš število 20. Po drugi strani doživiš skrajni neuspeh, če vržeš število 1. Naj opozorimo, da nimajo vsi preizkusi posebnih dodatnih učinkov ob skrajnem uspehu ali neuspehu, zato je treba take rezultate obravnavati kot običajne uspehe oziroma neuspehe.

Podajmo primer: tvoj lik zasleduje poškodovano himero. Nenadoma njegovo pot zapre deroča reka. Odločiš se, da jo boš preplaval/a, vendar Vodja igre to označi za tvegano dejanje in te zaprosi za preizkus veščine Atletika (ker plavanje spada pod Atletiko). Na svojem pustolovskem obrazcu prebereš, da ima tvoj lik za take preizkuse popravek +8. Ko vržeš k20, pade 18, s prištetim popravkom pa to znese 26. Vodja igre to vrednost primerja s težavnostnim razredom (ki je 16) in ugotovi, da gre za skrajni uspeh (ker je tvoj rezultat presegel TR za 10). Tvoj lik urno preplava na drugi breg reke in nadaljuje lov za himero, sicer moker, a nepoškodovan. Če bi bil končni rezultat preizkusa manjši od 26, vendar večji ali enak 16, bi tvoj lik preplaval pol poti čez reko. Če bi bil rezultat manjši od 16, bi tvoj lik odneslo s tokom ali, še huje, potegnilo v globino, kjer bi se pričel utapljati!

Takšni preizkusi so srce igre in jih ves čas opravljamo, da določimo izide opravil. Čeprav je met kocke ključen, števila, ki jih prištejemo metu (imenovana \textbf{popravki}), pogosto ločijo uspeh od neuspeha. Vsak lik ima mnogo metrik, ki uravnavajo, v čem je lik dober. Vsako metriko sestavljata ustrezen popravek sposobnosti in \textbf{izvedenski} prištevek, poleg tega pa jo lahko spremenijo še drugi faktorji, denimo prištevki ali odbitki, ki jih doprinesejo oprema, uroki, izučitve, čarobni predmeti in druge posebne okoliščine.

Izvedenstvo je preprost način za določanje likove splošne izurjenosti in primernosti za dano opravilo. Delimo ga na pet nivojev: \textbf{neizurjen}, \textbf{izurjen}, \textbf{strokovnjak}, \textbf{mojster} in \textbf{legendaren}. Vsak nivo nudi svoj izvedenski prištevek. Če si v metriki neizurjen, je tvoj izvedenski prištevek enak +0 -- zanesti se moraš zgolj na popravek sposobnosti. Če je tvoj izvedenski nivo kateri od ostalih štirih, je tvoj prištevek enak stopnji tvojega lika plus število, ki ga določa tvoj izvedenski nivo (2 za izurjenega, 4 za strokovnjaka, 6 za mojstra in 8 za legendarnega). Izvedenski nivoji so del skoraj vsake metrike v igri.

\subsection{Raziskovanje}
Večino časa bo tvoj lik raziskoval svet, se sporazumeval z drugimi bitji, potoval sem in tja ter premagoval izzive. To je raziskovalni del igre. Raziskovanje nima hudih omejitev in igralci prosto posegajo v pripoved, kadar želijo kaj posebnega storiti, na primer odjahati iz mesta na konjih, slediti plemenu plenilskih orkov, se izogibati njihovim izvidnikom in prepričati lokalnega lovca, da pomaga v prihajajočem boju.

V tem igralnem segmentu Vodja sprašuje igralce, kaj njihovi liki počno, medtem ko raziskujejo. To je važno v primeru, da pride do spopada, saj lahko dejanja likov vplivajo na to, kako se spopad prične in v kakšnem položaju so takrat igralni liki.

\subsection{Naleti}
Med pustolovščino včasih preprost preizkus veščine ne bo dovolj, da rešiš težavo. Ko pred tvojim likom stoji grozljiva pošast, ni druge možnosti, kot da se ji postavi po robu. V igri Stezosledca temu rečemo nalet. Le-ta običajno zajema spopad, vendar se lahko uporabi tudi v drugih situacijah, kjer je čas ključnega pomena, na primer med begom ali izmikanjem pastem.

Medtem ko je raziskovanje precej proste narave, so naleti bolj strukturirani. Igralci in Vodja igre mečejo kocke za \textbf{pobudo}, s katero določijo, po kakšnem vrstnem redu vsi sodelujoči ukrepajo. Nalet sestavlja več \textbf{krogov}, ki v svetu igre trajajo po šest sekund. V enem krogu vsak sodelujoči pride na \textbf{potezo}. Ko si na potezi, lahko izvedeš do tri \textbf{dejanja}. Večina preprostih opravil, kot so zavihteti orožje, narediti krajši premik po bojnem polju, odpreti vrata ali zamahniti z mečem, porabijo eno dejanje. Obstajajo tudi \textbf{dejavnosti}, ki porabijo več kot eno dejanje -- to so pogosto posebne zmožnosti, značilne za razred in izučitve tvojega lika. En običajen primer dejavnosti je čaranje uroka, ki večinoma porabi dve dejanji.

\textbf{Zanemarljiva dejanja}, recimo da izpustimo predmet iz rok, se ne štejejo med tri dejanja, ki jih lahko izvedeš na potezi. Med potezo lahko lik opravi še en \textbf{odziv}. To je posebna vrsta dejanja, ki ga lahko tvoj lik izvede tudi takrat, ko nisi na potezi, vendar le v določenih okoliščinah in če imaš sposobnost, ki ti to omogoča. Faloti imajo denimo na razpolago izučitev, s katero se lahko v odziv izmaknejo sovražnikovemu napadu.

Najpogosteje v spopadu \textbf{napademo} drugo bitje. Za to moramo opraviti preizkus, ki mora preseči \textbf{zaščitni razred (ZR)} bitja, ki ga napadamo. Napad lahko izvedemo z orožjem, urokom ali celo deli telesa, denimo s pestjo, kremplji ali repom. Metu prištejemo popravek glede na to, v kolikšni meri smo izvedeni v vrsti napada, ki ga izberemo, glede na naše ocene sposobnosti in glede na druge prištevke ali odbitke v dani situaciji. Tarčin ZR poračunamo iz izvedenskega nivoja za oklep, ki ga nosi, in njenega popravka Gibčnosti. Če napad uspe, tarča utrpi poškodbe, in če dosežemo skrajni uspeh, se količina poškodb podvoji!

Na svoji potezi lahko napadeš večkrat, a vsak naslednji napad po prvem je bolj negotov. To predstavlja \textbf{odbitek za več napadov}, ki pri drugem napadu znaša -5, že pri tretjem pa zrase na -10. Ta odbitek je mogoče omiliti na več načinov in po koncu tvoje poteze se ponastavi.

Če proti tvojemu liku poleti \textit{čaroben blisk} ali strašni beli zmaj buhne vanj svojo ledeno sapo, boš moral/a opraviti rešilni met, ki predstavlja sposobnost tvojega lika, da se izogne nevarnosti ali kako drugače vzdrži napad na svoje telo ali dušo. Rešilni met je preizkus, ki ga opravimo proti TR-ju uroka ali posebne sposobnosti, ki poskuša škodovati tvojemu liku. Obstajajo tri vrste rešilnih metov, in če je lik v kateri izveden, to veliko pove o njegovi trpežnosti. Rešilni met \textbf{trdnosti} uporabimo, kadar sta napadu podvržena likovo zdravje in življenjska energija. Rešilni met \textbf{refleksov} je potreben, ko se mora lik izmakniti nevarnosti. To je običajno nekaj, kar vpliva na širšo okolico, denimo žgoč sunek čarobne ognjene krogle. Nazadnje omenimo še rešilni met \textbf{volje}, s katerim se branimo pred uroki in učinki, ki škodujejo umu -- taka sta na primer urok očarljivosti ali urok zmede. Kadar je rešilni met uspešen, omili škodljiv učinek, pri skrajnem uspehu pa jo običajno lik odnese nepoškodovan.

Napadi, uroki, nevarnosti in posebne sposobnosti pogosto povzročijo bodisi \textbf{poškodbe} bodisi \textbf{stanja} -- včasih pa oboje. Poškodbene točke odbijamo od bitjevih \textbf{življenjskih točk (ŽT)} -- te merijo zdravje --, in ko bitje izgubi vse življenjske točke, pade v nezavest in lahko tudi umre! Spopad traja, dokler ena stran ne porazi nasprotnikov. To lahko pomeni, da le-ti pobegnejo ali se predajo, zelo pogosto pa poraz pomeni, da so mrtvi ali še umirajo. Stanja lahko bitju začasno omejijo dejanja, ki jih lahko stori, in njegove preizkuse bremenijo z odbitki. Nekatera stanja so celo trajna in se jih je mogoče rešiti le z mogočno čarovnijo.

\subsection{Prosti čas}
Liki niso ves čas na pustolovščinah. Sem in tja se morajo odpočiti, si zaceliti rane, načrtovati bodoče podvige ali hoditi po nakupih. V Stezosledcu temu rečemo prosti čas. V tem segmentu čas teče hitreje, medtem ko liki opravljajo svoja dolgoročna opravila in cilje. V prostem času se lahko liki ukvarjajo s trgovanjem in tako zaslužijo nekaj denarja, tisti z ustreznimi veščinami pa se lahko polotijo kakšne obrti, si izdelajo novo opremo ali čarobne predmete. Liki lahko izkoristijo prosti čas za ponovno urjenje in spremenijo svojo razvojno smer. Lahko se lotijo miselnega preiskovanja kakega problema, se naučijo novih urokov ali celo vodijo podjetje ali kraljestvo!

\subsection{Ključni pojmi}

Preden ustvariš svoj prvi lik ali pustolovščino, se spoznaj s prgiščem pomembnejših izrazov. Nekaj si jih spoznal/a že na prejšnjih straneh.

\newcommand{\rpgterm}[3]{%
    \begin{description}[leftmargin=!, labelwidth=20pt]
        \item[#1] \textit{(#2)} #3
    \end{description}}

\rpgterm{Ocene sposobnosti}{Ability Scores}{Vsako bitje ima šest ocen sposobnosti: Moč, Gibčnost, Kosntitucijo, Inteligenco, Modrost in Karizmo. Te ocene predstavljajo bitjevo dispozicijo in osnovne značilnosti. Višja je ocena, večja je bitjeva dovzetnost za to sposobnost. Več o ocenah sposobnosti piše kasneje v tem poglavju.}

\rpgterm{Opredelitev}{Alignment}{Opredelitev predstavlja bitjeve temeljne moralne in etične vrednote.}

\rpgterm{Rod}{Ancestry}{Rod je širša družina, ki ji lik pripada. Rod določa likove začetne življenjske točke, jezike, čute in hitrost, poleg tega pa liku podari tudi rodovne izučitve. O rodovih več piše v drugem poglavju.}

\rpgterm{Zaščitni razred (ZR)}{Armor Class (AC)}{Vsa bitja v igri imajo zaščitni razred. Ta ocenjuje, kako težko je zadeti in poškodovati bitje. Zaščitni razred napadenega bitja predstavlja težavnostni razred za uspešnost napada.}

\rpgterm{Napad}{Attack}{Kadar neko bitje poskusi škoditi drugemu bitju, izvede napad. Večinoma napade izvedemo z orožji, včasih pa napademo tudi s pestmi, zadržimo ali potisnemo tarčo z rokami ali pa napademo z urokom.}

\rpgterm{Ozadje}{Background}{Ozadje zajema vse, kar je lik doživel, preden je postal pustolovec. Vsako ozadje ti podari eno izučitev in izurjenost v eni ali več veščinah. Več o ozadjih piše v drugem poglavju.}

\rpgterm{Prištevki in odbitki}{Bonuses, Penalties}{Prištevki in odbitki se uporabljajo pri preizkusih in določenih metrikah. Obstaja več vrst prištevkov oziroma odbitkov. Če prejmeš več prištevkov iste vrste, uporabiš le najvišjega. Enako velja z odbitki.}

\rpgterm{Razred}{Class}{Razred predstavlja pustolovsko stroko, ki jo lik izbere. Njegov razred določa večino njegovih izvedenstev, na vsaki novi stopnji podeli liku več življenjskih točk in omogoči nabor razrednih izučitev. O razredih govori tretje poglavje.}

\rpgterm{Stanje}{Condition}{Dlje časa trajajoči učinek, ki omeji likovo obnašanje ali spreminja njegove sposobnosti, se imenuje stanje. Pravila o temeljnih stanjih v igri najdeš v dodatku z naslovom Stanja na koncu knjige.}

\rpgterm{Valuta}{Currency}{Najpogostejši denarni enoti v igri sta zlatnik (zl) in srebrnik (sr). En zlatnik je vreden 10 srebrnikov. Poleg tega je 1 srebrnik vreden 10 bakrencev (bk), 10 zlatnikov pa lahko zamenjamo za 1 platino (pl). Liki pričnejo igro s 15 zlatniki (oz. 150 srebrniki) za opremo.}

\rpgterm{Posebna izučitev}{Feat}{Posebne izučitve svojemu liku izbereš prek rodu, ozadja, razreda, splošnega urjenja ali piljenja veščine. Nekatere izučitve to omogočajo uporabo posebnih dejanj.}

\rpgterm{Vodja igre}{Game Master}{Vodja igre je igralec, ki sodi igro ter opisuje zgodbo in svet, ki ga ostali igralci raziskujejo.}

\rpgterm{Golarion}{Golarion}{Stezosledec se dogaja na planetu Golarion v Dobi razgubljenih znamenj. To je starodaven svet, poln raznolikih ljudstev in kultur, razburljvih krajev in smrtonosnih hudobnežev. Več o Dobi razgubljenih znamenj, Golarionu in njegovih božanstvih piše v osmem poglavju.}

\rpgterm{Življenjske točke (ŽT)}{Hit Points (HP)}{Življenjske točke predstavljajo količino žalega, ki ga lahko bitje utrpi, preden izgubi zavest in začne umirati. Poškodbe skozi poškodbene točke odbijajo življenjske točke, možno pa si jih je tudi povrniti skozi zdravljenje.}

\rpgterm{Pobuda}{Initiative}{Na začetku spopada vsa bitja mečejo kocke, s katerimi določijo svojo pobudo. Ta predstavlja vrstni red, v katerem bitja ukrepajo tekom vsakega kroga spopada. Višji kot je rezultat meta, prej bo bitje na vrsti v krogu. Pobuda in spopad sta opisana v devetem poglavju.}

\rpgterm{Stopnja}{Level}{Stopnja je številka, ki meri posameznikovo skupno moč. Stopnja igranih likov seže od prve do dvajsete in predstavlja njihovo stopnjo izkušenosti. Stopnje pošasti, neigranih likov, nevarnosti, bolezni in strupov sežejo od -1 do 30, pomenijo pa jakost njihove nevarnosti. Tudi predmeti imajo stopnjo, večinoma med nič in dvajset, včasih pa tudi več, ta stopnja pa pove, kako močni so in kako primerni so za zaklad.\\Stopnje urokov so na intervalu od ena do deset in merijo njihovo moč. Liki in pošasti lahko često pričarajo le določeno število urokov na posamezni stopnji.}

\rpgterm{Neigrani lik}{Nonplayer Character (NPC)}{Neigrane like krmili Vodja igre. Ti liki se sporazumevajo z igralci in pomagajo razvijati pripoved.}

\rpgterm{Zaznavanje}{Perception}{Zaznavanje meri likovo sposobnost zaznati skrite predmete ali nenavadne razmere. Poleg tega običajno določi, kako hitro lik plane v akcijo med spopadom. Opisano je v devetem poglavju.}

\rpgterm{Igrani lik}{Player Character (PC)}{To je lik, ki ga ustvari in igra igralec.}

\rpgterm{Izvedenstvo}{Proficiency}{Izvedenstvo je sistem, ki meri likovo obvladovanje določenega opravila ali sposobnost. Ima pet nivojev: neizurjen, izurjen, strokovnjak, mojster in legendaren. Izvedenstvo ti podeli prištevek, ki ga prišteješ k naslednjim popravkom in metrikam: zaščitni razred, napadalni meti, Zaznavanje, rešilni meti, veščine in učinkovitost urokov. Če si neizurjen, je tvoj izvedenski prištevek +0. Če si izurjen, strokovnjak, mojster ali legendaren, tvoj izvedenski prištevek šteje 2, 4, 6 in 8, v tem vrstnem redu.}

\rpgterm{Redkost}{Rarity}{Nekateri elementi igre imajo oznako redkosti, s katero povemo, kako pogosti so v svetu igre. Redkost v prvi vrsti povezujemo z opremo in čarobnimi predmeti, vendar imajo stopnjo redkosti tudi uroki, izučitve in druga pravila. Če med značilnostmi elementa ne najdemo oznake redkosti, je element privzeto običajen. Do neobičajnih predmetov pridejo le posamezniki s posebnim znanjem, iz določene kulture ali iz ustreznega dela sveta. Redkih predmetov skoraj ni mogoče najti in jih običajno razdaja le Vodja igre, edinstveni predmeti pa so, kot že ime pove, edini svoje vrste v vsej igri. Vodja igre lahko spremeni sistem redkosti ali redkost posameznih predmetov, da se prilegajo zgodbi, ki jo želi povedati.}

\rpgterm{Igranje vlog}{Roleplaying}{Opisovanju likovih dejanj, pogosto skozi uprizarjanje lika, rečemo igranje vloge. Takrat igralec govori in opisuje dejanje z vidika svojega lika.}

\rpgterm{Krog}{Round}{Krog je časovno obdobje, v katerem vsi sodelujoči dobijo priložnost za ukrepanje. V svetu igre en krog pomeni približno šest sekund.}

\rpgterm{Rešilni met}{Saving Throw (Save)}{Ko se bitje sooči z nevarnostjo, izvede rešilni met, s katerim poskuša ublažiti posledice. Rešilni met ti privzeto pripada, zanj ni treba porabiti dejanja ali odziva. Za razliko od večine preizkusov rešilni met izvede neposredna ali posredna tarča škodljivega dejanja tako, da vrže k20, bitje, ki je škodljivo dejanje povzročilo, pa postavi TR. Rešilne mete delimo na tri skupine: rešilne mete trdnosti (proti boleznim, strupom in telesnim učinkom), refleksov (proti učinkom, ki se jim lahko lik hitro izmakne) in rešilne mete volje (proti vplivom na um in osebnost).}

\rpgterm{Veščina}{Skill}{Veščina predstavlja zmožnost bitja, da izvaja naloge, za katere sta potrebni vaja in metodičnost. Veščine so v celoti opisane v četrtem poglavju. Vsako veščino lahko v neki meri koristijo tudi tisti, ki v njej niso izurjeni, za nekatere vrste uporabe pa je potrebna izurjenost.}

\rpgterm{Hitrost}{Speed}{Hitrost je količina poti, ki jo lik 
lahko opravi v času enega samega dejanja. Merimo jo v metrih.}

\rpgterm{Urok}{Spell}{Uroki so čarobni učinki, ki jih udejanjimo z mističnimi izreki in kretnjami, poznanimi le posebej izurjenim ali prirojeno nadarjenim. Čaranje uroka je dejavnost, ki navadno terja dve dejanji. Vsak urok ima tarčo, postopek pričaranja, učinek in načine, kako se mu upreti. Če imaš v okviru razreda dostop do urokov, o njihovi uporabi podrobno piše v opisu dotičnega razreda v tretjem poglavju. Uroki sami so zapisani v sedmem poglavju.}

\rpgterm{Značilnost}{Trait}{Z značilnostjo naslavljamo vse dodatne informacije o elementu v pravilih, kot sta denimo šola čaranja in redkost. Značilnost pogosto pojasnjuje, kako ostala pravila vplivajo na sposobnost, bitje, predmet ali drug element, ki premore to značilnost.}

\rpgterm{Poteza}{Turn}{Tekom kroga vsako bitje pride na potezo. Med potezo lahko praviloma izvede tri dejanja.}

\section{Primeri igre}

V nadaljevanju ti bomo poskušali na primeru predstaviti okvirno idejo igranja Stezosledca. Vodja igre je Erik. Tilka igra pogumnega človeškega borca Valerosa, Jan igra smrtonosno vilinsko falotinjo Merisiel in Julija predstavlja Kyro, ognjevito človeško klerikinjo, ki služi boginji Sarenri. Druščina je pravkar premagala hordo povratnikov (mrtvecev, ki so se prebudili v življenje) in si zdaj utira pot v starodavni mavzolej.

\vspace{20pt}

\textbf{Erik:} Pred vami stoji vhod v grobnico. Razpadajoče stopnišče se spušča dol v temo. Izza vrat butne obupen smrad po gnilem mesu.

\textbf{Tilka:} Slab zadah me nič ne gane! Izvlečem svoj meč in pripravim ščit.

\textbf{Julija:} Sarenrina luč nas bo vodila. S svojo versko podobo pričaram urok svetlobe.

\textbf{Erik:} Prav. Čez stopnice se razleze krasna svetloba. Videti je, da se po treh metrih spusta odprejo v dvorano. Vodne luže polnijo špranje med neenakomerno posajenimi kamnitimi kladami.

\textbf{Jan:} Prva bom šla, da preverim, če je varno. Izvlekla bom svoj rapir in se previdno spustila po stopnicah, pri tem pa pazila na kakršnekoli pasti.

\textbf{Erik:} Velja, ampak iskanje pasti je skrit preizkus, zato ga bom metal zate. Koliko šteje tvoj popravek Zaznavanja?

\textbf{Jan:} +5.

\vspace{20pt}

\textit{Erik za svojim voditeljskim zastorom, da drugi ne morejo videti, vrže k20 in dobi 17. S popravkom vred je rezultat 22, kar je več kot dovolj, da lik zagleda sprožilno žico na tretji stopnici.}

\vspace{20pt}

\textbf{Erik:} Tvoja previdnost se izplača! Tik nad tretjo stopnico, na višini gležnja, opaziš tenko žico.

\textbf{Jan:} Nanjo opozorim svoja tovariša in se podam naprej.

\textbf{Tilka:} Hodim tik za Merisiel. Previdno stopim čez žico in oprezam za nevarnostmi.

\textbf{Julija:} Tudi jaz.

\textbf{Erik:} Dobro! Po stopnicah pridete v grobnico. Po sobi so razporejene starodavne lesene krste, pokrite s pajčevinami in prahom. Na dvignjenem podiju pred vami je kamnita krsta, v katero so vrezani zlovešči simboli. Vidite lahko, da je bila nekoč ovita v železne verige, vendar zdaj povsod po sobi ležijo njeni zviti ostanki, skupaj s kamnitimi razbitinami, ki so verjetno ostale od pokrova krste. Glede na škodo presodite, da je bil razbit od znotraj!

\textbf{Julija:} Sarenra nas obvarji. Izvlečem svoj meč in stopim naprej. Želim si bolje ogledati tiste simbole.

\textbf{Tilka:} Pospremil jo bom. Tole mi ni čisto nič všeč.

\textbf{Jan:} Mislim, da bom raje ostala zadaj in se skrila za eno od drugih krst.

\textbf{Erik:} Merisiel se skrije, medtem ko vidva napredujeta. Bližje kot sta podiju, hujši je smrad po gnilobi. Nazadnje je že skoraj neznosen. Nenadoma zagledata izvor ogabnega smradu. Iz krste se dvigne grozljivo mrtvo bitje. Nekoč je bilo morda človek, vendar je iz njegovega uvelega trupla težko razbrati. Meso ima modre barve, kot sveža podpluta, in tako močno napeto čez kosti, da je na mestih natrgano. Plešasto je, zašiljenih ušes, najhuje pa je, da mu iz ust zevajo drobni ostri zobje in gnusen, dolg jezik.

\textbf{Tilka:} Najbrž ni prijatelj?

\textbf{Erik:} Vse prej kot to. Pripravlja se, da vaju bo napadlo. Vrzite pobudo! Valeros in Kyra mečeta z Zaznavanjem, Merisiel pa s Tiholazenjem.

\vspace{20pt}
\textit{Vsi vržejo pobudo. Tilka za Valerosa vrže 2, skupaj s popravki 8. Julija za Kyro vrže nekoliko bolje, končni rezultat je 14. Jan za pobudo uporabi Tiholazenje, ker se je ob pričetku spopada Merisiel skrivala. Izid je 17, skupaj s popravkom pa kar 25! Erik vrže pobudo za povratniško bitje in kocka pristane na 12. Erik si vse te pobude zapiše in like razvrsti po njej od najvišje do najnižje.}
\vspace{20pt}

\textbf{Erik:} Merisiel je prva na vrsti. Precej prepričana si, da te pošast ni opazila.

\textbf{Jan:} Super! Za svoje prvo dejanje bi rada izvlekla bodalo. Za svoje drugo dejanje bi se rada pomaknila bližje.

\textbf{Erik:} Z enim dejanjem se lahko pošasti približaš na 4,5 metra.

\textbf{Jan:} Perfektno. Za konec bom vanjo vrgla svoje bodalo!

\vspace{20pt}
\textit{Jan vrže 13 in prišteje 8, ker je Merisiel vešča lučanja bodal, kar skupaj znese 21. A zaradi oddaljenosti od tarče utrpi odbitek -2, torej je končni rezultat 19. Erik v svojih zapiskih prebere, da ima pošast zaščitni razred 18.}
\vspace{20pt}

\textbf{Erik:} Zadeneš! Vrzi še za poškodbene točke.

\textbf{Jan:} Velja. Ker napadam prikrito, lahko zadam dodatne poškodbe.

\vspace{20pt}
\textit{Faloti lahko sovražnikom, ki v spopadu niso še niti enkrat prišli na vrsto, povzročijo dodatne poškodbe. To velja tudi za sovražnike, ki niso pozorni na napadajočega falota. Jan vrže 1k4 za bodalo in 1k6 za prikriti napad, nato pa prišteje še 4 za Merisielino Gibčnost, kar na koncu vrne 9.}
\vspace{20pt}

\textbf{Erik:} Bitje od bolečine zasika, ko se mu bodalo zarije v ramo, vendar ga to ne omaja. Jan, porabil si vsa dejanja. Nadaljuje Kyra!

\textbf{Julija:} Mislim, da je to povratnik. Kaj vem o njem?

\textbf{Erik:} Za eno dejanje se lahko spomniš svojega izobraževanja o živih mrtvecih. Preveri svoje Veroslovje.

\vspace{20pt}
\textit{Julija vrže 16 in prišteje Kyrin popravek +8 pri veščini Veroslovje. Končen rezultat je 24.}
\vspace{20pt}

\textbf{Erik:} Sprva pomisliš, da bi lahko bil ghul -- povratnik, ki se prehranjuje z mesom mrtvih --, vendar hud smrad razkrije resnico. Pošast pred vami je zlopuh, močnejša sorta ghula. Skoraj gotova si, da ti je lahko od njegovega smradu slabo in da te lahko z dotikom krempljev ohromi.

\textbf{Julija:} Slabo. Za svoji zadnji dejanji bom pričarala blagoslov. Tako vsak v moji neposredni bližini prejme prištevek +1 na napadalne mete.

\vspace{20pt}
\textit{Čaranje uroka porabi dve dejanji in zahteva dva dejavnika: telesni in glasovni. Telesni dejavnik so zapletene kretnje, s katerimi se urok pričara, glasovni dejavnik pa so Kyrine molitve k svojemu božanstvu.}
\vspace{20pt}

\textbf{Erik:} Dobro! Zlopuh skoči iz krste proti Merisiel. Od blizu je njegov telesni smrad neznosen. Izvedi rešilni met trdnosti!

\vspace{20pt}
\textit{Jan vrže 8, skupaj 14.}
\vspace{20pt}

\textbf{Erik:} Ne bo dovolj. Končaš v stanju slabosti prve stopnje, zato boš za večino svojih metov k20 prejel odbitek -1. Zdaj bitje skoči nate in šavsne po tebi!

\textbf{Jan:} O, o! Z odzivom se izmaknem.

\vspace{20pt}
\textit{Erik izvede napadalni met za zlopuha, pade 9. Preveri pošastine metrike in prišteje 11. Tako ima na koncu 20. Merisielin zaščitni razred bi bil normalno 19, a ji izučitev Urni izmik med tem enim napadom poveča ZR za 2. V tem primeru to pokvari zlopuhov napad.}
\vspace{20pt}

\textbf{Erik:} Te 20 zadene?

\textbf{Jan:} Ne, za las zgreši!

\textbf{Erik:} Izmakneš se proč, medtem ko zlopuhov sluzasti jezik oplazi tvoj oklep. Za svoje zadnje dejanje te hudoba napade s svojimi kremplji.

\vspace{20pt}
\textit{Erik vrže za drugi napad, tokrat s kremplji. Navadno bi ta napad utrpel odbitek -5, ker je že drugi napad po vrsti, toda ker ima pošast spretne kremplje, je odbitek le -4. Kocka pristane na 19, temu prišteje 11, kar je zlopuhov napadalni popravek, in odšteje 4. Tako na koncu dobi 26.}
\vspace{20pt}

\textbf{Erik:} Izognila si se zlopuhovemu ugrizu, toda njegov koščen krempelj ti zareže čez obraz!

\vspace{20pt}
\textit{Erik ve, da napad zadene, in vrže kocko za poškodbene točke, ki znesejo 8.}
\vspace{20pt}

\textbf{Erik:} Utrpela si 8 poškodbenih točk. Opravi rešilni met trdnosti, saj okoli svoje rane začenjaš čutiti odrevenelost.

\vspace{20pt}
\textit{Jan opravi rešilni met trdnosti. Kocka vrne 4, s svojim prištevkom in odbitkom zaradi stanja slabosti pa Jan na koncu ostane s pičlo devetico.}
\vspace{20pt}

\textbf{Jan:} Danes nimam sreče. Najbrž 9 ne bo dovolj?

\textbf{Erik:} Bojim se, da ne. Omrtvičena si!

\vspace{20pt}
\textit{Erik si zabeleži, da je Merisiel omrtvičena in zato ne more ukrepati, vendar bo na koncu vsake svoje poteze dobila možnost opraviti rešilni met in se rešiti odrevenelosti.}
\vspace{20pt}

\textbf{Erik:} Iz zlopuhovih skrivenčenih ust se izvije suh, pokljajoč smeh, toda njegove poteze je konec. Valeros, še ti si na potezi.

\textbf{Tilka:} Še pravi čas! Dvignem svoj ščit in uporabim preostali dejanji za to, da izvedem Nenadni naskok!

\vspace{20pt}
\textit{Nenadni naskok je izučitev, značilna za borce, zaradi katere se lahko Valeros premika z dvakratno hitrostjo in po koncu premika še napade -- vse to za ceno le dveh dejanj.}
\vspace{20pt}

\textbf{Erik:} Ko se približaš, se ti grozni smrad upre. Opravi rešilni met trdnosti.

\vspace{20pt}
\textit{Tilka rešilni met opravi z 19.}
\vspace{20pt}

\textbf{Erik:} Nekako pretolčeš obupni vonj. Izvedi napad.

\vspace{20pt}
\textit{Tilka vrže kocko in pade 20.}
\vspace{20pt}

\textbf{Tilka:} 20 imam! To bo skrajni uspeh!

\textbf{Erik:} Tvoje rezilo zaseka naravnost v bitjev vrat. To povzroči dvojno škodo!

\vspace{20pt}
\textit{Tilka vrže k8 -- izid je 5 -- in prišteje 4 zaradi Valerosovega popravka Moči. Ker je dosegla skrajni uspeh, rezultat podvoji.}
\vspace{20pt}

\textbf{Tilka:} Kar 18 točk! To ga gotovo ubije!

\textbf{Erik:} Žal ne. Iz vratu mu mezi črna sluz, vendar se zgolj obrne proti tebi. Iz oči mu puhti sovraštvo!

\textbf{Tilka:} Ojej.

\vspace{20pt}
\textit{Tako se zaključi prvi krog spopada. Drugi krog se začne takoj zatem in zanj se uporabi isti vrstni red pobude kot v prvem krogu. Boja še zdaleč ni konec ...}
\vspace{20pt}

\section{Uporaba knjige}
Prvo poglavje je namenjeno temu, da te nauči osnov igre, preostanek knjige pa je urejen tako, da ga je lažje uporabljati kot vir informacij med samo igro. Pravila se združujejo v poglavja. Prvih nekaj poglavij se osredotoča na ustvarjanje likov, zadnji dve pa vsebujeta pravila in nasvete za Vodje iger ter pester nabor zakladov. Spodaj so krajši povzetki posameznih poglavij.

\subsubsection{Prvo poglavje: Uvod}
Uvod je narejen tako, da ti pomaga razumeti osnove Stezosledca. Vsebuje tudi pravila za izgradnjo in višanje stopenj likov. Zaključi se z zgledom, kako ustvariti lik prve stopnje.

\subsubsection{Drugo poglavje: Rodovi in ozadja}
V tem poglavju najdeš pravila za najobičajnejše rodove v Dobi razgubljenih znamenj, vključno z njihovimi posebnimi izučitvami. Na koncu poglavja so navedena ozadja in razdelek o jezikih, saj ti pogosto vplivajo na tvojo izbiro rodu.

\subsubsection{Tretje poglavje: Razredi}
To poglavje popisuje pravila za vseh dvanajst razredov. Pri vsakem razredu so navedene smernice za igranje tega razreda, pravila za izgradnjo in napredovanje lika iz tega razreda, primeri likov ter vse razredne posebne izučitve, ki jih imajo člani razreda na voljo. V tem poglavju so tudi pravila za živalske sopotnike in zaupnice, ki jih lahko imajo pripadniki različnih razredov. Na koncu poglavja so še pravila za arhetipe -- posebne možnosti, ki so na voljo likom, ko napredujejo po stopnjah. Po teh pravilih se lahko lik ukvarja tudi s sposobnostmi iz drugega razreda ali koncepta.

\subsubsection{Četrto poglavje: Veščine}
Tukaj so predstavljena pravila za uporabljanje veščin. Opisujejo, kaj lahko lik počne s posamezno veščino glede na svoj izvedenski nivo. Rod, ozadje in razred definirajo nekatere izmed likovih izvedenstev in vsak lik lahko izbere še dodatne veščine, ki odražajo njegovo osebnost in izurjenost.

\subsubsection{Peto poglavje: Posebne izučitve}
Ko lik napreduje po stopnjah, prejme nove posebne izučitve, ki predstavljajo njegovo krepitev veščin. Poglavje predstavi splošne izučitve in veščinske izučitve (ki so podmnožica splošnih izučitev).

\subsubsection{Šesto poglavje: Oprema}
V tem poglavju najdemo oklepe, orožja in drugo opremo, cenik storitev, bivanja in živali (na primer konjev, psov in črednih živali).

\subsubsection{Sedmo poglavje: Uroki}
To poglavje se začne s pravili za čaranje urokov, določanje njihovih učinkov in odbijanje nasprotnikovih urokov. Sledijo seznami urokov, urejeni po stopnjah, in nato še pravila za vsak urok posebej. Potem so navedeni še področni uroki -- posebni uroki, ki jih podeljujejo posamezne razredne sposobnosti in izučitve. Čeprav se večina urokov pojavi na več različnih uročnih seznamih, so področni uroki prihranjeni za člane določenih razredov, zato so združeni glede na razred, da jih lažje najdemo. Na koncu poglavja se nahajajo še pravila za obrede ter zahtevnejše in tvegane uroke, ki jih lahko pričara katerikoli lik.

\subsubsection{Osmo poglavje: Doba razgubljenih znamenj}
V tem poglavju so na kratko opisani svet Golariona, ljudstva, ki živijo v njem, in časovnica pomembnejših dogodkov. Liki, ki častijo kako božanstvo, tukaj najdejo pravila v povezavi s svojo vero.

\subsubsection{Deveto poglavje: Igranje igre}
To poglavje je pomembno, ker vsebuje univerzalna pravila za igranje Stezosledca, vključno s pravili različnih segmentov igre, osnovnih dejanj likov, spopada in umiranja. S tem poglavjem naj se pobližje spozna vsak igralec, še zlasti Vodja igre.

\subsubsection{Deseto poglavje: Vodenje igre}
Tu tiči kup nasvetov in smernic, ki pomagajo Vodjem igre prirediti zanimivo zgodbo ter vzpostaviti zabavno in spodbudno igralno vzdušje, v katerem lahko vsak igralec ustvari tak lik, kakršnega si želi igrati. V tem poglavju so tudi za Vodjo igre pomembna pravila, recimo pravila za pasti, okoljske nevarnosti in druge tegobe (na primer prekletstva, bolezni in strupe) ter smernice za postavljanje TR-jev in nagrajevanje igranih likov.

\subsubsection{Enajsto poglavje: Zakladi}
Zakladi, ki jih liki odkrijejo tekom svojih dogodivščin, so številnih oblik, od zlata in rubinov do mogočnih čarobnih predmetov. V tem poglavju so smernice za deljenje zakladov likom in opisi na stotine čarovnih predmetov. Tukaj so tudi pravila za alkimijske predmete.

\subsubsection{Dodatki}
Na koncu knjige se skriva dodatek s pravili za vsa stanja, na katera lahko naletiš med igro. Tam sta tudi neizpolnjen primerek pustolovskega obrazca in slovarček pojmov, s katerimi se srečujemo med igro.

\section{Zapis pravil}
V knjigi boš opazil/a določene zapise in formatiranje, ki se ti bodo sprva zdeli nekoliko čudni. Določena pravila in elementi bodo zapisani z veliko začetnico in v poševni pisavi. Tako jih lažje prepoznamo v besedilu.

Nekatere metrike, veščine, izučitve, dejanja in določeni elementi igre so zapisani z veliko začetnico. Če nekje piše „pod Gibčnost sodi tudi Akrobatika“, takoj prepoznamo, da gre za veščino in sposobnost v okviru pravil. V tem slovenskem prevodu je sicer takih primerov manj kot v angleščini (npr. v primeru zaščitnega razreda (\textit{Armor Class})).

Če je beseda ali besedna zveza zapisana poševno, predstavlja urok ali čaroben predmet. Ko vidiš stavek „Vrata so zapečatena z \textit{zaklepom},“ je torej jasno, da v tem primeru beseda pomeni urok \textit{zaklep}, ne dejansko zapiralo.

Stezosledec včasih uporablja tudi okrajšave, denimo ZR za zaščitni razred, TR za težavnostni razred in ŽT za življenjske točke. Če te kakšna okrajšava kdaj zmede, jo lahko poiščeš v slovarčku in kazalu izrazja na koncu knjige.

\phantomsection
\subsection{Razumevanje dejanj}
Liki in njihovi nasprotniki vplivajo na svet okoli sebe z izvajanjem dejanj. To še posebej drži med spopadi, kjer šteje prav vsako dejanje. Ko izvedeš dejanje, na nekaj učinkuješ. Ta učinek je lahko samoumeven, včasih pa je za izvedbo dejanja potreben met kocke in je učinek odvisen od izida na kocki.

Dejanja bodo v knjigi zaznamovale posebne sličice.

\subsubsection{\oneaction Posamezna dejanja}
Posamezna dejanja označujemo s simbolom \oneaction. Podnje štejemo najpreprostejša, najpogostejša dejanja. V spopadu lahko v eni potezi izvedeš tri posamezna dejanja v kakršnemkoli vrstnem redu.

\subsubsection{\reaction Odzivi}
Odzive označujemo s simbolom \reaction. Ta dejanja lahko izvedeš tudi takrat, ko nisi na vrsti. Vsak krog imaš na voljo en odziv in unovčiš ga lahko le, ko se izpolni določen pogoj. Pogosto je to dejanje nekega drugega bitja.

\subsubsection{\freeaction Zanemarljiva dejanja}
Za zanemarljiva dejanja uporabljamo simbol \freeaction. Zanje ti ni treba zapraviti nobenega posameznega dejanja ali odziva. Zanemarljiva dejanja lahko sproži nek dogodek, podobno kot odzive. V takih primerih jih lahko obravnavaš kot odzive -- tudi ko nisi na potezi. A na posamezen sprožilni dogodek lahko izvedeš le eno zanemarljivo dejanje. Če imaš torej na nek dogodek vezanih več zanemarljivih dejanj, se moraš odločiti, katerega boš izvedel/a. Če zanemarljivo dejanje ni vezano na sprožilec, ga lahko obravnavaš kot posamezno dejanje, ki pa se ne šteje med tri dejanja na tvoji potezi.

\subsubsection{Dejavnosti}
Dejavnosti so posebna opravila, ki jih izvedeš tako, da kombiniraš eno ali več dejanj. Dejavnost običajno stane dve ali več dejanj in ti omogoča storiti nekaj več, kot bi ti dopustilo posamezno dejanje. Da dejavnost obrodi nek rezultat, moraš porabiti vsa dejanja, ki jih zahteva. Ena najpogostejših dejavnosti je gotovo čaranje, saj je za večino urokov potrebno več kot eno dejanje.

Dejavnosti, ki porabijo dve dejanji, označujemo s simbolom \twoactions, dejavnosti, ki porabijo tri dejanja, pa s simbolom \threeactions. Nekaj posebnih dejavnosti, na primer uroke, ki jih lahko pričaramo v trenutku, lahko izvedemo z zanemarljivim dejanjem ali odzivom.

Dejavnosti so tudi vsa opravila, ki trajajo dlje od ene poteze. Če je dejavnost mogoče opravljati med raziskovanjem, ima raziskovalno značilnost. Dejavnost, ki traja en dan ali dlje in jo je mogoče opravljati le med prostim časom, ima prostočasno značilnost.

\subsection{Branje pravil}
Knjiga vsebuje na stotine igralnih sestavin, ki likom omogočajo, da se na dogodke v igri odzovejo na zanimive in izvirne načine. Vsi liki lahko koristijo temeljna dejanja iz devetega poglavja, za posamezen lik pa pogosto veljajo še posebna pravila, po katerih lahko počno še kakšne stvari, ki jih drugi liki ne morejo. Večina teh pravil je izučitev, pridobljenih ob stvarjenju lika ali ob napredovanju po stopnjah.

Igralni elementi so vedno predstavljeni v obliki okenca lastnosti, ki povzema vsa potrebna pravila za igranje pošasti, lika, predmeta ali drugega elementa. Okenca lastnosti so včasih opremljena z razlago formata, kjer je to potrebno. Na primer, v razdelku o rodovih v drugem poglavju so pravila za vsakega od osnovnih šestih rodov in na začetku poglavja se nahaja razlaga teh pravil.

Če določeno polje v okencu lastnosti ni veljavno, je iz okenca izpuščeno, zato nimajo vsi elementi vseh polj, ki so navedena spodaj. Dejanja, odzive in zanemarljiva dejanja ločimo po simbolih, ki se nahajajo poleg njihovih imen. Dejavnost, ki jo je mogoče opraviti v roku ene poteze, ima simbol, ki označuje, koliko dejanj je potrebnih zanjo. Dejavnosti, ki trajajo dlje časa, teh simbolov ne prikazujejo. Če mora lik doseči določeno stopnjo, preden lahko koristi neko sposobnost, je ta stopnja navedena desno od imena v okencu lastnosti. Pravila imajo pogosto posebne značilnosti (le-te so navedene v slovarčku in kazalu izrazja).

Uroki, alkimijski predmeti in čarobni predmeti uporabljajo podoben format, vendar njihova okenca vsebujejo precej posebnih polj (več o urokih piše v sedmem poglavju, o alkimijskih in čarobnih predmetih pa v enajstem poglavju).

\section{Izdelava likov}
Če nisi Vodja igra, pri igri Stezosledca najprej ustvariš svoj lik. Sam/a si izmisliš, kaj je tvoj lik počel v preteklosti ter kakšno osebnost in pogled na svet ima. Ti predstavljajo temelje za igranje vlog med igro.